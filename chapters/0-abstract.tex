\begin{abstract}
The principle of least privilege serves as an essential guideline in designing secure computing systems. However, implementing this in real-world systems through various programming languages has proved to be difficult and has allowed for many vulnerabilities in privilege escalation. One proposed solution is to have capability-based security primitives in programming languages for modules and objects. A capability is a unique token that provides the authority to perform a specific set of actions on a selected resource. However, its effectiveness as a language design choice for real-world applications remains to be seen.

To answer this question, we designed a comparative study to compare programmer productivity, security of the designs, and extensibility of packages in capability-based module systems vs.\ others. Our main goal was to determine whether module systems/packages having capabilities from the ground up provide usability and security advantages compared to their absence. The study used two programming languages - one with object capabilities (Wyvern) and the other with support for capabilities via external libraries (Rust).

Preliminary findings show that programs designed in Wyvern provided higher security guarantees in some cases, and users found using object capabilities an easy-to-use secure abstraction layer for managing critical resources. However, a lack of tooling for showing appropriate errors or code completion introduced challenges in writing code. Hence, future work requires a more in-depth study to define and validate current user-centric methods of designing capability-based languages. Further work also involves classifying security vulnerabilities solved by capabilities and building the necessary tools in Wyvern to make capabilities more viable as a design choice in programming languages.

% TODO
\keywords{Object Capabilities \sep Language-based Security \sep Module Systems \sep Rust \sep Wyvern}
\end{abstract}

