\section{Concluding Remarks}\label{sec-conclusion}

\subsection{Conclusion}

\noindent
We conducted a comparative study of traditional versus capability-based module systems by interviewing four upper-level undergraduate students. In doing so, we identified two main hypotheses relating to the usability and security of programs in capability-based languages to help programmers be productive and motivate further work in researching further secure systems. % to improve

We hope these hypotheses will improve the understanding of reasons to design secure programming languages via capability-based language systems. Furthermore, if principles from the study are used to provide better tooling for existing capability-based languages, we hope that it will have broad benefits in terms of writing better software in terms of programmer productivity, security, and extensibility.

\subsection{Future Work}

\noindent
Current results show that we have only scratched the surface in this problem domain. Further avenues for research mainly include improving the design of the study based on the following:

\noindent
\textbf{Larger target audience} Preferably working professionals who are domain experts with at least one of the programming languages being surveyed. We found that our study needed quantitative analysis and collect more anecdotal evidence to support our claims. The current sample size is small for doing any of the two. This would also help us use more complex software design problems during interviews.  

% https://doi.org/10.14236%2Fewic%2FHCI2008.9
\noindent
\textbf{Including more modern programming languages} For a more comprehensive comparison and seeing what features are needed in capability-based designed languages to make it more usable for general programmers without sacrificing security. The current study has only two programming languages with widely different syntaxes, so having different studies with closer languages would provide a better benchmarks in evaluating usability.

\noindent
\textbf{Finding more Security Vulnerabilities} The study designs should be based on existing CVE vulnerabilities. Vulnerabilities classified in CVEs are based on large-scale software, where decisions are made on an architectural level. We can design studies about employing capabilities patterns on larger systems to achieve this.

% \textbf{Grounded Theory Study} - Validating/coming up with hypothesis. - Maybe thematic analysis is not enough here

% inference for SML